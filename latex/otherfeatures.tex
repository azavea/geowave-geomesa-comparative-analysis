\subsection{Other features}
\label{sec:featurecompare:other}

This section gives a summary of features that are either found in one project and not the other,
or are found in both projects with considerable differences.


\subsubsection{Features Found in GeoWave and not in GeoMesa}
\label{sec:featurecompare:other:wave}

\begin{itemize}
\item Integration with Mapnik
\item Integration with PDAL for reading and writing point cloud data from/to GeoWave
\item Time interval queries: The ability to index data that exists within an interval of time, and query for intersecting intervals
\item Pixel-level subsampling: A visualization performance optimization that returns a subset of the resulting data based on pixel width
\item DBScan clustering
\end{itemize}


\subsubsection{Features Found in GeoMesa and not in GeoWave}
\label{sec:featurecompare:other:mesa}

\begin{itemize}
\item Tube selection analysis: Given a collection of points (with associated times), return a similar collections of points in terms of where the lines connecting said points exist
\item Loose bounding box queries: Return all points where the index matches the space filling curve index query, and skip secondary filtering
\item JSON configurable ingest tooling
\item Cassandra backend (alpha support)
\item Google BigTable support (marked experimental)
\item Kafka GeoTools DataStore
\item Cost-based query optimization
\item Querying for a subset of \texttt{SimpleFeature} attributes and only returning the necessary data
\end{itemize}


\subsubsection{Raster Support}
\label{sec:featurecompare:other:raster}

Both projects have some level of raster support; however, GeoWave's raster support appears to be more mature than GeoMesa's.
According to GeoMesa's documentation, rasters must be pre-tiled to a specific size, projection, have non-overlapping extents, and must only single band rasters.
GeoWave supports multiband rasters, and includes tiling and merging strategies that allow you to ingest rasters that are not pre-tiled.
While GeoWave's raster support is more mature than GeoMesa's, both project's support of raster data is not entirely mature; for instance,
there is no support for anything but spatial rasters (i.e. you cannot ingest spatiotemporal raster data such as timestamped imagery).

  
\subsubsection{HBase Backend}
\label{sec:featurecompare:other:hbase}

Both projects have support for an HBase backend; however, GeoWave's support for HBase is more mature.
The GeoWave development team has expressed the amount of work that has gone into trying to match the performance of the HBase backend to that of their Accumulo backend.
The GeoMesa team expressed that there has not yet been an equal level of effort to achieve relative parity between their Accumulo and HBase backends.


\subsubsection{Attribute/Secondary indexing}
\label{sec:featurecompare:other:secondary}

GeoMesa and GeoWave both have features around secondary indexing that are quire different.
A detailed comparison of those features for both GeoMesa and GeoWave can be found in Appendix \ref{appendix:features}: Details of GeoMesa and GeoWave features.
